\documentclass{article}
\usepackage{amsmath}
\begin{document}

\title{Building polynomials with roots $r^k$, $k\perp n$}
\author{A. Kruppa}
\maketitle


\begin{abstract}
This note presents some of the basic ideas underlying the scheme for building 
a reciprocal Laurent polynomial from its roots as described by Montgomery 
\cite{Montgomery2007}
\end{abstract}

\section{The set of integers coprime to $n$}
Let $S_n = \{1 \leq k < n, k \perp n\}$ be the set of positive integers 
coprime to and less than $n$, i.e. the smallest positive representatives of 
the units of $Z/Zn$.

Let $A + B$ denote the set of sums, $A + B = \{a+b, a\in A, b\in B\}$.

We now have, for $l \perp m$, 
\begin{displaymath}
S_{lm} \equiv lS_m + mS_l.
\end{displaymath}
The sets $S_{lm}$ and $lS_m + mS_l$ are not identical as $lS_m + mS_l$ does 
generally not contain only integers $<n$. However, each residue class coprime
to $n$ is represented by exactly one integer in $lS_m + mS_l$. 
Since $n-1 \in S_n$, 
$a \in lS_m + mS_l \Rightarrow 0 < a \leq 2lm - l - m$.

%Or more generally, if $n = \prod_{i=1}^{k} l_i$, $l_j \perp l_i$ for 
%$j \neq i$, then $S_n = \sum_{i=1}^{k} n/l_jS_{l_j}$.

By definition of the Euler totient function $\varphi$, $|S_n| = \varphi(n)$.

\section{A polynomial with roots $r^k$, $k\perp n$}
Let
\begin{displaymath}
F_{n,r}(x) = \prod_{k \in S_n} (x-r^k).
\end{displaymath}

Then, with $p \perp n$,
\begin{eqnarray*}
F_{pn,r}(x) & = & \prod_{k \in S_{pn}} \left(x-r^k\right) \\
            & = & \prod_{k \in pS_n + nS_p} \left(x-r^k\right) \\
            & = & \prod_{i \in nS_p} \prod_{j \in pS_n} \left(x-r^{i+j}\right) \\
            & = & \prod_{i \in nS_p} \prod_{j \in S_n} \left(x-r^{i+pj}\right) \\
            & = & \prod_{i \in nS_p} \prod_{j \in S_n} \left(r^i \left(\frac{x}{r^i}-r^{pj}\right)\right) \\
            & = & \prod_{i \in nS_p} r^{i\varphi(n)} \prod_{j \in S_n} \left(\frac{x}{r^i}-r^{pj}\right) \\
            & = & \prod_{i \in nS_p} r^{i\varphi(n)} F_{n,r^p}\left(\frac{x}{r^i}\right)
\end{eqnarray*}

If $F_{n,r}(x) = \sum_{k=0}^{\varphi(n)} f_k x^k$, then 
\begin{eqnarray*}
r^{i\varphi(n)} F_{n,r}\left(\frac{x}{r^i}\right) & = &\\
\sum_{k=0}^{\varphi(n)} r^{i\varphi(n)} f_k x^k r^{-ik} & = & \\
\sum_{k=0}^{\varphi(n)} f_k x^k r^{i(\varphi(n)-k)} &&
\end{eqnarray*}
so we can generate $r^{i\varphi(n)} F_{n,r}\left(\frac{x}{r^i}\right)$ from 
$F_{n,r}(x)$ by multiplying the coefficients from highest to lowest by powers 
of $r$ in an increasing geometric progression, using $2$ multiplications 
per coefficient. This means that to produce $F_{pn,r}(x)$, we need to compute 
$F_{n,r^p}(x)$ only once.

\begin{thebibliography}{}
\bibitem{Montgomery2007} Peter-Lawrence Montgomery: 
{\it Record Factorizations (I Hope!) Using P-1/FFT. Draft February 8, 2007}. 
Unpublished manuscript.
\end{thebibliography}

\end{document}

