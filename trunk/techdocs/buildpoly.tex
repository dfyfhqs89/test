\documentclass{article}
\usepackage{amsmath,amssymb}
\begin{document}

\title{Building polynomials with roots $r^k$, $k\perp n$}
\author{A. Kruppa}
\maketitle


\begin{abstract}
This note presents some of the basic ideas underlying the scheme for building 
a reciprocal Laurent polynomial from its roots as described by Montgomery 
\cite{Montgomery2007}.
\end{abstract}


\section{The set of integers coprime to $n$}
Let $S_n = \{k + n\mathbb{Z}, k \perp n\}$
\footnote{The notation $a\perp b$ for ``$a$ is relatively prime to $b$''
          follows the suggestion in \cite[4.5]{Graham_Knuth_Patashnik}} 
be the set of residue classes coprime to $n$.

Let $A + B$ denote the set of sums, $A + B = \{a+b, a\in A, b\in B\}$.

Then we have, for $l \perp m$, 
\begin{equation}\label{S_sum}
S_{lm} = lS_m + mS_l.
\end{equation}

By definition of the Euler totient function $\varphi$, $|S_n| = \varphi(n)$.

\subsection{Positive representatives}
Let $\bar{S}_n = \{1 \leq k < n, k \perp n\}$ be the set of smallest 
positive integers coprime to $n$ and less than $n$, i.e. the smallest 
positive representatives of the units of $\mathbb{Z}/\mathbb{Z}n$. 
Unfortunately, (\ref{S_sum}) does not immediately carry over to 
$\bar{S}_{lm}$, as $l\bar{S}_m + m\bar{S}_l$ contains integers $> lm$.
Since $n-1 \in \bar{S}_n$, we get
$l(m-1) + m(l-1) = 2lm-l-m > lm-1$ for $l,m > 1$. For example, 
$\bar{S}_2 = \{1\}$, $\bar{S}_3 = \{1,2\}$, and 
$3\bar{S}_2 + 2\bar{S}_3 = \{5, 7\} \neq \{1, 5\} = \bar{S}_6$. With 
$1, n-1 \in \bar{S}_n$, we have 
$a \in l\bar{S}_m + m\bar{S}_l \Rightarrow l+m \leq a \leq 2lm - l - m$. 
Hence, $l\bar{S}_m + m\bar{S}_l$ lies on an interval of length 
$2(lm - l - m)$.

\subsection{Representatives symmetric around $0$}
Let $\hat{S}_n = \{|k| \leq (n-1)/2, k \perp n\}$ be the 
set of integers of smallest absolute value that are coprime to $n$, 
for $n > 2$. As for $\bar{S}_n$, (\ref{S_sum}) does not carry over, 
as for example $3\hat{S}_2 + 2\hat{S}_3 = \{-1,7\} \neq \{-1,1\}$.
For odd $l,m,n$, since $-\frac{n-1}{2},\frac{n-1}{2} \in \hat{S}_n$,
$a \in l\hat{S}_m + m\hat{S}_l \Rightarrow |a| \leq lm-\frac{l+m}{2}$, 
covering an interval of length $2lm - l - m$.

\pagebreak[1]
For $n \equiv 3 \pmod{4}$, $\hat{S}_n$ can be factored by 
$r=\frac{n+1}{4}$, $\hat{S}_n = \{ -r, r\} + \{-r+1, \ldots, r-1\}.$
The elements of the second set form an arithmetic progression and 
so it can be factored again if its cardinality is composite.

Montgomery suggests sets with elements symmetric around $0$ and odd, 
i.e. $\tilde{S}_n = \{2i-n, 1\leq i \leq n-1 \}$. The advantage is 
that the elements of these sets form arithmetic progressions which 
can always be factored as a set of sums, if the cardinality of the 
set is composite. The factors are again arithmetic progressions, so
$\tilde{S}_n$ can always be factored into sets of prime cardinality.
The disadvantage is that $\tilde{S}_n$ covers an interval of length 
$2n-4$, noticably larger than $n-2$ for $\bar{S}_n$ or $n-1$ for 
$\hat{S}_n$.

\subsection{Combining several sets of sums}

For composite $n$, we can write
\begin{equation}\label{S_longsum}
S_n \equiv \sum_{p\mid n} \frac{n}{p}S_p.
\end{equation}
If the absolute values of the elements of $S_p$ are bounded by $cp$, 
we clearly have absolute values of the elements of the RHS of 
(\ref{S_longsum}) bounded by $cn\nu(n)$, where $\nu(n)$ is the 
number of prime divisors in $n$. 
This bound grows only linearly in $c$, so choosing, i.e., $\hat{S}_n$ 
over $\tilde{S}_n$ has only a relatively small impact on the length 
of the interval covered by the set of sums.

\section{A polynomial with roots $r^k$, $k\perp n$}
In the following, $S_n$ represents any of $\bar{S}_n$, $\hat{S}_n$, 
$\tilde{S}_n$ or similar choice of particular representatives of
residue classes coprime to $n$.

Let
\begin{displaymath}
F_{n,r}(x) = \prod_{k \in S_n} (x-r^k).
\end{displaymath}

Then, with $p \perp n$,
\begin{eqnarray*}
F_{pn,r}(x) & = & \prod_{k \in S_{pn}} \left(x-r^k\right) \\
            & = & \prod_{k \in pS_n + nS_p} \left(x-r^k\right) \\
            & = & \prod_{i \in nS_p} \prod_{j \in pS_n} \left(x-r^{i+j}\right) \\
            & = & \prod_{i \in nS_p} \prod_{j \in S_n} \left(x-r^{i+pj}\right) \\
            & = & \prod_{i \in nS_p} \prod_{j \in S_n} \left(r^i \left(\frac{x}{r^i}-r^{pj}\right)\right) \\
            & = & \prod_{i \in nS_p} r^{i\varphi(n)} \prod_{j \in S_n} \left(\frac{x}{r^i}-r^{pj}\right) \\
            & = & \prod_{i \in nS_p} r^{i\varphi(n)} F_{n,r^p}\left(\frac{x}{r^i}\right)
\end{eqnarray*}

If $F_{n,r}(x) = \sum_{k=0}^{\varphi(n)} f_k x^k$, then 
\begin{eqnarray*}
r^{i\varphi(n)} F_{n,r}\left(\frac{x}{r^i}\right) & = &\\
\sum_{k=0}^{\varphi(n)} r^{i\varphi(n)} f_k x^k r^{-ik} & = & \\
\sum_{k=0}^{\varphi(n)} f_k x^k r^{i(\varphi(n)-k)} &&
\end{eqnarray*}
so we can generate $r^{i\varphi(n)} F_{n,r}\left(\frac{x}{r^i}\right)$ from 
$F_{n,r}(x)$ by multiplying the coefficients from highest to lowest by powers 
of $r$ in an increasing geometric progression, using $2$ multiplications 
per coefficient. This means that to produce $F_{pn,r}(x)$, we need to compute 
$F_{n,r^p}(x)$ only once.

\begin{thebibliography}{}
\bibitem{Montgomery2007} Peter-Lawrence Montgomery: 
{\it Record Factorizations (I Hope!) Using P-1/FFT. Draft February 8, 2007}. 
Unpublished manuscript.
\bibitem{Graham_Knuth_Patashnik}Graham, Knuth, and Patashnik: 
{\it Concrete Mathematics}. Second edition. 1989. Addison Wesley
\end{thebibliography}

\end{document}

